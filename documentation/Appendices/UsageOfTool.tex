\chapter{Usage of Tool}
\textbf{Tool usage:} \newline
\mintinline{bash}{pgsynthdata [OPTIONS]... DBNAMEIN [DBNAMEGEN]} \\
\newline
\section{Options}
\textbf{Tool options:}
\begin{itemize}
	\item \mintinline{bash}{DBNAMEGEN} - Name of the database to be created
	\item \mintinline{bash}{-show/--show} - Shows database stats (default)
	\item \mintinline{bash}{-generate/--generate} - Generates new synthetic data to database \textit{DBNAMEGEN}
	\item \mintinline{bash}{-mf/--mf} - Multiplication factor for thegenerated synthetic data (default: 1.0)
	\item \mintinline{bash}{-tables/--tables} - Name(s) of table(s) to be filled, separated with ',', ignoring other tables (default: fill all tables)
	\item \mintinline{bash}{-r/--recreate} - (Re-)create new DBNAMEGEN and schema (default: don't recreate database/schema, just truncate the tables)
	\item \mintinline{bash}{-O/--owner} - Owner of new database (default: same as user)
	\item \mintinline{bash}{-v/--version} - Show version information, then quit
	\item \mintinline{bash}{-h/--help} - Show tool help, then quit
	\newline
\end{itemize}
\textbf{Connection options:}
\begin{itemize}
	\item \mintinline{bash}{DBNAMEIN} - Name of the existing database to connect to
	\item \mintinline{bash}{-H/--hostname} - Name of the PostgreSQL server (default: \textit{localhost})
	\item \mintinline{bash}{-P/--port} - Port of the PostgreSQL server (default: \textit{5432})
	\item \mintinline{bash}{-U/--user} - PostgreSQL server username
	\newline
\end{itemize}
\section{Usage Examples}
\textbf{Some usage examples:}
\begin{itemize}
	\item \mintinline{bash}{python pgsynthdata.py test postgres -show}
	\begin{itemize}
		\item Connects to database \textit{test}, host=\textit{localhost}, port=\textit{5432}, default user with password \textit{postgres}
		\item Shows statistics of the database \textit{test}
	\end{itemize}
	\item \mintinline{bash}{python pgsynthdata.py db pw1234 -H myHost -p 8070 -U testuser -show}
	\begin{itemize}
		\item Connects to database \textit{db}, host=\textit{myHost}, port=\textit{8070}, user=\textit{testuser} with password \textit{pw1234}
		\item Shows statistics of the database \textit{db}
	\end{itemize}
	\item \mintinline[breaklines]{bash}{python pgsynthdata.py dbin dbgen pw1234 -H myHost -U testuser -generate}
	\begin{itemize}
		\item Connects to database \textit{dbin}, host=\textit{myHost}, port=\textit{5432}, user=\textit{testuser} with password \textit{pw1234}
		\item Truncates tables of \textit{dbgen} and generates synthetic data into them
	\end{itemize}
	\item \mintinline{bash}{python pgsynthdata.py dbin dbgencreate pw123 -U myUser -generate -r}
	\begin{itemize}
		\item Connects to database \textit{dbin}, host=\textit{localhost}, port=\textit{5432}, user=\textit{myUser} with password \textit{pw123}
		\item Creates new database \textit{dbgencreate} with the same schema as \textit{dbin} and generates synthetic data into it
	\end{itemize}
	\item \mintinline[breaklines]{bash}{python pgsynthdata.py dbin dbgencreate pw123 -U myUser -generate -tables myTable1, myTable2}
	\begin{itemize}
		\item Connects to database \textit{dbin}, host=\textit{localhost}, port=\textit{5432}, user=\textit{myUser} with password \textit{pw123}
		\item Only truncates the \textit{myTable1} and \textit{myTable2} tables and generates synthetic into them
	\end{itemize}
	\item \mintinline{bash}{python pgsynthdata.py --version}
	\begin{itemize}
		\item Shows the version of this tool
	\end{itemize}
	\item \mintinline{bash}{python pgsynthdata.py --help}
	\begin{itemize}
		\item Shows the help information for this tool
	\end{itemize}
\end{itemize}